%**************************************************************
% file contenente le impostazioni della tesi
%**************************************************************

%**************************************************************
% Frontespizio
%**************************************************************

% Autore
\newcommand{\myName}{Federico Zanardo}                                    
\newcommand{\myTitle}{Analysis of the implementation of the Stipula legal calculus using Distributed Ledger Technologies}

% Tipo di tesi                   
\newcommand{\myDegree}{Master's degree thesis}

% Università             
\newcommand{\myUni}{University of Padua}

% Facoltà       
\newcommand{\myFaculty}{Master degree in Computer Science}

% Dipartimento
\newcommand{\myDepartment}{Department of Mathematics "Tullio Levi-Civita"}

% Titolo del relatore
\newcommand{\profTitle}{Professor }

% Relatore
\newcommand{\myProf}{Silvia Crafa}

% Luogo
\newcommand{\myLocation}{Padova}

% Anno accademico
\newcommand{\myAA}{2022-2023}

% Data discussione
\newcommand{\myTime}{21 April 2023}

\newcommand\textbox[1]{%
	\parbox{.333\textwidth}{#1}%
}

%Numerazione equazione
\newcommand\numberthis{\addtocounter{equation}{1}\tag{\theequation}}

%Argmin
\DeclareMathOperator*{\argminA}{arg\,min} % Jan Hlavacek
\DeclareMathOperator*{\argminB}{argmin}   % Jan Hlavacek
\DeclareMathOperator*{\argminC}{\arg\min}   % rbp

\newcommand{\argminD}{\arg\!\min} % AlfC

\newcommand{\argminE}{\mathop{\mathrm{argmin}}}          % ASdeL
\newcommand{\argminF}{\mathop{\mathrm{argmin}}\limits}   % ASdeL

% limits on side
\DeclareMathOperator{\argminG}{arg\,min} % Jan Hlavacek
\DeclareMathOperator{\argminH}{argmin}   % Jan Hlavacek
\newcommand{\argminI}{\mathop{\mathrm{argmin}}\nolimits} % ASdeL

%Argmax
\DeclareMathOperator*{\argmaxA}{arg\,max} % Jan Hlavacek
\DeclareMathOperator*{\argmaxB}{argmax}   % Jan Hlavacek
\DeclareMathOperator*{\argmaxC}{\arg\max}   % rbp

\newcommand{\argmaxD}{\arg\!\max} % AlfC

\newcommand{\argmaxE}{\mathop{\mathrm{argmax}}}          % ASdeL
\newcommand{\argmaxF}{\mathop{\mathrm{argmax}}\limits}   % ASdeL

% limits on side
\DeclareMathOperator{\argmaxG}{arg\,max} % Jan Hlavacek
\DeclareMathOperator{\argmaxH}{argmax}   % Jan Hlavacek
\newcommand{\argmaxI}{\mathop{\mathrm{argmax}}\nolimits} % ASdeL

\newcommand{\cs}[1]{\texttt{\symbol{`\\}#1}}


%**************************************************************
% Impostazioni di impaginazione
% see: http://wwwcdf.pd.infn.it/AppuntiLinux/a2547.htm
%**************************************************************

\setlength{\parindent}{14pt}   % larghezza rientro della prima riga
\setlength{\parskip}{0pt}   % distanza tra i paragrafi


%**************************************************************
% Impostazioni di biblatex
%**************************************************************



\bibliography{bibliografia.bib} % database di biblatex 


\defbibheading{bibliography} {
    \cleardoublepage
    \phantomsection 
    \addcontentsline{toc}{chapter}{\bibname}
    \chapter*{\bibname\markboth{\bibname}{\bibname}}
}

\setlength\bibitemsep{1.5\itemsep} % spazio tra entry

\DeclareBibliographyCategory{paper}
\DeclareBibliographyCategory{web}

\addtocategory{paper}{Statistical-learning-R}
\addtocategory{paper}{Statistical-learning-theory}
\addtocategory{paper}{Machine-Learning-Mitchell}
\addtocategory{paper}{site:IEEE-Paper-UV-radionation-GHI}
\addtocategory{paper}{site:DNI-DHI-GHI-Cloudiness-correlation}
\addtocategory{paper}{Adaptive-Lasso-Paper}
\addtocategory{paper}{Elastic-Net-Paper}
\addtocategory{paper}{Skew-Paper}

\addtocategory{web}{site:Kippzonen-PV-application}
\addtocategory{web}{site:UV-effects-on-humans}
\addtocategory{web}{site:NSRDB-official-site}
\addtocategory{web}{site:NSRDB-history}
\addtocategory{web}{site:NSRDB-processing-report}
\addtocategory{web}{site:NSRDB-old-repository}
\addtocategory{web}{site:NSRDB-new-repository}
\addtocategory{web}{site:NSRDB-content-presentation}
\addtocategory{web}{site:glmnet-library}
\addtocategory{web}{site:leaps-library}

\defbibheading{paper}{\section*{}}
\defbibheading{web}{\section*{}}


%**************************************************************
% Impostazioni di caption
%**************************************************************
\captionsetup{
    tableposition=top,
    figureposition=bottom,
    font=small,
    format=hang,
    labelfont=bf
}

%**************************************************************
% Impostazioni di glossaries
%**************************************************************
\makeglossaries

%**************************************************************
% Acronimi
%**************************************************************
\renewcommand{\acronymname}{Acronimi e abbreviazioni}

\newacronym[description={\glslink{apig}{Application Program Interface}}]
{api}{API}{Application Program Interface}

\newacronym[description={\glslink{erpg}{Enterprise Resource Planning}}]
{erp}{ERP}{Enterprise Resource Planning}

\newacronym[description={\glslink{ibmg}{International Business Machines Corporation}}]
{ibm}{IBM}{International Business Machines Corporation}

\newacronym[description={\glslink{nlug}{Natural Language Understanding}}]
{nlu}{NLU}{Natural Language Understanding}

\newacronym[description={\glslink{pocg}{Proof of Concept}}]
{poc}{PoC}{Proof of Concept}

\newacronym[description={\glslink{jvmg}{Java Virtual Machine}}]
{jvm}{JVM}{Java Virtual Machine}

\newacronym[description={\glslink{es6}{ECMAScript6}}]
{es6}{ES6}{}

\newacronym[description={\glslink{html}{HyperText Markup Language}}]
{html}{HTML}{}

\newacronym[description={\glslink{css}{Cascading Style Sheets}}]
{css}{CSS}{}

\newacronym[description={\glslink{php}{Hypertext Preprocessor}}]
{php}{PHP}{}

\newacronym[description={\glslink{ram}{Random Access Memory}}]
{ram}{RAM}{}

\newacronym[description={\glslink{sdk}{Software Development Kit}}]
{sdk}{SDK}{}

\newacronym[description={\glslink{http}{HyperText Transfer Protocol}}]
{http}{HTTP}{}
%**************************************************************
% Glossario
%**************************************************************
\renewcommand{\glossaryname}{Glossario}

\newglossaryentry{apig}
{
	name=\glslink{api}{API},
	text=API,
	sort=api,
	description={in informatica con il termine \emph{Application Programming Interface API} (ing. interfaccia di programmazione di un'applicazione) si indica un insieme di procedure disponibili al programmatore, di solito raggruppate, che formano un set di strumenti specifici per assolvere un determinato compito all'interno di un software. Lo scopo delle \emph{API} è ottenere un'astrazione, solitamente tra l'hardware e il programmatore o tra software a basso ed alto livello semplificando il lavoro di programmazione}
}


\newglossaryentry{erpg}
{
    name=\glslink{erp}{ERP},
    text=ERP,
    sort=erp,
    description={in informatica l'\emph{ERP, Enterprise Resource Planning} (ing. pianificazione delle risorse d'impresa), è un software di gestione che integra i processi di business rilevanti di un'azienda e le sue funzioni quali vendite, acquisti, gestione magazzino, finanza e contabilità. Integra quindi tutte le attività aziendali in un unico sistema che risulta essere indispensabile per supportare il Management}
}

\newglossaryentry{ibmg}
{
	name=\glslink{ibm}{IBM},
	text=IBM,
	sort=ibm,
	description={L'\emph{IBM, International Business Machines Corporation} è la più antica azienda nel mondo dell'informatica, ha sede negli Stati Uniti ed è tra le maggiori al mondo. Produce e commercializza hardware, software per computer, middleware e servizi informatici offrendo anche infrastrutture, servizi di hosting, cloud computing, intelligenza artificiale e consulenza nel settore informatico e strategico}
}

\newglossaryentry{nlug}
{
	name=\glslink{nlu}{NLU},
	text=NLU,
	sort=ibm,
	description={La \emph{NLU, Natural Language Understanding} (ing. comprensione del linguaggio naturale) è un software con intelligenza artificale capace di interpretare ed elaborare il linguaggio naturale}
}

\newglossaryentry{pocg}
{
	name=\glslink{pocg}{PoC},
	text=PoC,
	sort=poc,
	description={Con \emph{PoC, Proof of Concept} (ing. prova di fattibilità) si intende una realizzazione incompleta e abbozzata di un determinato progetto o applicativo allo scopo di provarne la fattibilità oppure dimostrare la fondatezza di alcuni principi o concetti costituenti. Un esempio tipico è quello di un prototipo}
}

\newglossaryentry{jvmg}
{
	name=\glslink{jvmg}{JVM},
	text=JVM,
	sort=jvm,
	description={La \emph{JVM, Java Virtual Machine} è il componente della piattaforma Java che esegue i programmi tradotti in bytecode dopo una prima fase di compilazione. Alcuni dei linguaggi di programmazione traducibili in bytecode sono Java, Kotlin e Scala}
}

\newglossaryentry{ttg}
{
	name=\glslink{ttg}{Test di Turing},
	text=Test di Turing,
	sort=Test di Turing,
	description={Il \emph{Test di Turing} è un criterio costruito dallo scienziato Alan Turing nel 1950 per determinare se una macchina sia in grado di comprendere il linguaggio naturale e più in generale di pensare}
}

\newglossaryentry{sdkg}
{
	name=\glslink{sdkg}{Software Development Kit},
	text=SDK,
	sort=Software Development Kit,
	description={In informatica un \emph{SDK, Software Development Kit} (ing. pacchetto di sviluppo software), indica un insieme di strumenti per lo sviluppo e la documentazione di software}
}

\newglossaryentry{httpg}
{
	name=\glslink{httpg}{HyperText Transfer Protocol},
	text=HTTP,
	sort=HTTP,
	description={In informatica \emph{HTTP, HyperText Transfer Protocol} (ing. protocollo di trasferimento di un ipertesto) è un protocollo a livello applicativo utilizzato come principale sistema di trasmissione delle informazioni sul Web. Esiste una versione detta \emph{HTTPS, HyperText Transfer Protocol over Secure Socket Layer} che implementa le stesse funzionalità applicando uno strato di crittografia}
}

\newglossaryentry{buildg}
{
	name=\glslink{buildg}{Build},
	text=build,
	sort=build,
	description={Nello sviluppo del software, \emph{build} indica il processo di trasformazione del codice sorgente in un artefatto eseguibile}
}

\newglossaryentry{firebaseg}
{
	name=\glslink{firebaseg}{Firebase},
	text=Firebase,
	sort=Firebase,
	description={\emph{Firebase} è una piattaforma di sviluppo per applicazioni Web e mobile sviluppata da Firebase Inc. nel 2011 e acquisita da Google nel 2014}
}

\newglossaryentry{authg}
{
	name=\glslink{authg}{OAuth 2.0},
	text=OAuth 2.0,
	sort=OAuth 2.0,
	description={\emph{OAuth 2.0} è un protocollo di rete aperto e standard, progettato specificamente per lavorare con il protocollo \emph{HTTP}. Consente l'emissione di un token d'accesso, da parte di un server che fornisce autorizzatizioni, verso un client, previa approvazione dell'utente proprietario della risorsa cui si intende accedere. Rispetto alla sua versione precedente (\emph{OAuth 1.0}) presenta una chiara divisione dei ruoli implementando un mediatore tra client e server}
}

\newglossaryentry{rldg}
{
	name=\glslink{rldg}{Railroad},
	text=Railroad,
	sort=railroad,
	description={I \emph{diagrammi railroad}, detti anche \emph{diagrammi di sintassi}, consistono in una rapprensentazione grafica per grammatiche libere da contesto}
}

\newglossaryentry{parsg}
{
	name=\glslink{parsg}{Parser},
	text=parser,
	sort=parser,
	description={In informatica il \emph{parser} è un software che realizza il parsing ovvero un processo di analisi sintattica. Più in dettaglio analizza un flusso continuo di dati in input e determina la correttezza della sua struttura grazie ad una grammatica formale}
}

\newglossaryentry{gramg}
{
	name=\glslink{gramg}{Grammatica},
	text=grammatica,
	sort=grammatica,
	description={Nella teoria dei linguaggi formali una \emph{grammatica} (detta anche grammatica formale) è una struttura astratta che descrive un linguaggio (formale) in modo preciso. È definita anche come sistema di regole che delineano matematicamente un insieme potenzilamente infinito di sequenze finite di simboli appartenenti ad un alfabeto finito}
}

\newglossaryentry{cfg}
{
	name=\glslink{cfg}{Grammatica libera da contesto},
	text=grammatica libera da contesto,
	sort=grammatica libera da contesto,
	description={Una \emph{grammatica libera da contesto} è una struttura astratta che descrive un linguaggio (formale) in modo preciso con una notazione naturalmente ricorsiva}
}

\newglossaryentry{brainstorming}
{
	name=\glslink{brainstorming}{Brainstorming},
	text=brainstorming,
	sort=brainstorming,
	description={Il \emph{brainstorming} è un metodo decisionale in cui la ricerca della soluzione di un dato problema è effettuata mediante sedute intensive di dibattito e confronto delle idee espresse liberamente dai partecipanti}
}
 % database di termini


%**************************************************************
% Impostazioni di graphicx
%**************************************************************
\graphicspath{{immagini/}} % cartella dove sono riposte le immagini


%**************************************************************
% Impostazioni di hyperref
%**************************************************************
\hypersetup{
    %hyperfootnotes=false,
    %pdfpagelabels,
    %draft,	% = elimina tutti i link (utile per stampe in bianco e nero)
    colorlinks=true,
    linktocpage=true,
    pdfstartpage=1,
    pdfstartview=FitV,
    % decommenta la riga seguente per avere link in nero (per esempio per la stampa in bianco e nero)
    %colorlinks=false, linktocpage=false, pdfborder={0 0 0}, pdfstartpage=1, pdfstartview=FitV,
    breaklinks=true,
    pdfpagemode=UseNone,
    pageanchor=true,
    pdfpagemode=UseOutlines,
    plainpages=false,
    bookmarksnumbered,
    bookmarksopen=true,
    bookmarksopenlevel=1,
    hypertexnames=true,
    pdfhighlight=/O,
    %nesting=true,
    %frenchlinks,
    %urlcolor=webbrown,
    %linkcolor=RoyalBlue,
    citecolor=RoyalBlue,
    urlcolor=Black,
    linkcolor=Black,
    %citecolor=Black, 
    %pagecolor=RoyalBlue,
    %urlcolor=Black, linkcolor=Black, citecolor=Black, %pagecolor=Black,
    pdftitle={\myTitle},
    pdfauthor={\textcopyright\ \myName, \myUni, \myFaculty},
    pdfsubject={},
    pdfkeywords={},
    pdfcreator={pdfLaTeX},
    pdfproducer={LaTeX}
}

%**************************************************************
% Impostazioni di itemize
%**************************************************************
\renewcommand{\labelitemi}{$\ast$}

%\renewcommand{\labelitemi}{$\bullet$}
%\renewcommand{\labelitemii}{$\cdot$}
%\renewcommand{\labelitemiii}{$\diamond$}
%\renewcommand{\labelitemiv}{$\ast$}


%**************************************************************
% Impostazioni di listings
%**************************************************************
\lstset{
    language=[LaTeX]Tex,%C++,
    keywordstyle=\color{Black}, %\bfseries, %colore RoyalBlue nel template
    basicstyle=\small\ttfamily,
    %identifierstyle=\color{NavyBlue},
    commentstyle=\color{Black}\ttfamily,
    stringstyle=\rmfamily,
    numbers=none, %left,%
    numberstyle=\scriptsize, %\tiny
    stepnumber=5,
    numbersep=8pt,
    showstringspaces=false,
    breaklines=true,
    frameround=ftff,
    frame=single
} 


%**************************************************************
% Impostazioni di xcolor
%**************************************************************
\definecolor{webgreen}{rgb}{0,.5,0}
\definecolor{webbrown}{rgb}{.6,0,0}


%**************************************************************
% Altro
%**************************************************************

\newcommand{\omissis}{[\dots\negthinspace]} % produce [...]

% eccezioni all'algoritmo di sillabazione
\hyphenation
{
    ma-cro-istru-zio-ne
    gi-ral-din
}

\newcommand{\sectionname}{sezione}
\addto\captionsitalian{\renewcommand{\figurename}{Figura}
                       \renewcommand{\tablename}{Tabella}}

\newcommand{\glsfirstoccur}{\ap{{[g]}}}

\newcommand{\intro}[1]{\emph{\textsf{#1}}}

%**************************************************************
% Environment per ``rischi''
%**************************************************************
\newcounter{riskcounter}                % define a counter
\setcounter{riskcounter}{0}             % set the counter to some initial value

%%%% Parameters
% #1: Title
\newenvironment{risk}[1]{
    \refstepcounter{riskcounter}        % increment counter
    \par \noindent                      % start new paragraph
    \textbf{\arabic{riskcounter}. #1}   % display the title before the 
                                        % content of the environment is displayed 
}{
    \par\medskip
}

\newcommand{\riskname}{Rischio}

\newcommand{\riskdescription}[1]{\textbf{\\Descrizione:} #1.}

\newcommand{\risksolution}[1]{\textbf{\\Soluzione:} #1.}

%**************************************************************
% Environment per ``use case''
%**************************************************************
\newcounter{usecasecounter}             % define a counter
\setcounter{usecasecounter}{0}          % set the counter to some initial value

%%%% Parameters
% #1: ID
% #2: Nome
\newenvironment{usecase}[2]{
    \renewcommand{\theusecasecounter}{\usecasename #1}  % this is where the display of 
                                                        % the counter is overwritten/modified
    \refstepcounter{usecasecounter}             % increment counter
    \vspace{10pt}
    \par \noindent                              % start new paragraph
    {\large \textbf{\usecasename #1: #2}}       % display the title before the 
                                                % content of the environment is displayed 
    \medskip
}{
    \medskip
}

\newcommand{\usecasename}{UC}

\newcommand{\usecaseactors}[1]{\textbf{\\Attori Principali:} #1. \vspace{4pt}}
\newcommand{\usecasepre}[1]{\textbf{\\Precondizioni:} #1. \vspace{4pt}}
\newcommand{\usecasedesc}[1]{\textbf{\\Descrizione:} #1. \vspace{4pt}}
\newcommand{\usecasepost}[1]{\textbf{\\Postcondizioni:} #1. \vspace{4pt}}
\newcommand{\usecasealt}[1]{\textbf{\\Scenario Alternativo:} #1. \vspace{4pt}}

%**************************************************************
% Environment per ``namespace description''
%**************************************************************

\newenvironment{namespacedesc}{
    \vspace{10pt}
    \par \noindent                              % start new paragraph
    \begin{description} 
}{
    \end{description}
    \medskip
}

\newcommand{\classdesc}[2]{\item[\textbf{#1:}] #2}