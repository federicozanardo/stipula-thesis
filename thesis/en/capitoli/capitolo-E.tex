% !TEX encoding = UTF-8
% !TEX TS-program = pdflatex
% !TEX root = ../tesi.tex

%**************************************************************
\chapter{Docker}
\label{app:docker}
%**************************************************************

\section{Dockerfile}

\begin{Verbatim}[numbers=left,xleftmargin=1cm,firstnumber=1,breaklines=true,tabsize=2]
	# Setup Gradle
	FROM gradle:7.6.0-jdk8 AS TEMP_BUILD_IMAGE
	ENV APP_HOME=/usr/app
	WORKDIR $APP_HOME
	COPY build.gradle settings.gradle $APP_HOME
	
	COPY gradle $APP_HOME/gradle
	COPY --chown=gradle:gradle . /home/gradle/src
	USER root
	RUN chown -R gradle /home/gradle/src
	
	RUN gradle build || return 0
	COPY . .
	RUN gradle clean build
	
	# Setup Java
	FROM amazoncorretto:8
	ENV ARTIFACT_NAME=stipula-node-1.0-SNAPSHOT.jar
	ENV APP_HOME=/usr/app
	
	WORKDIR $APP_HOME
	COPY --from=TEMP_BUILD_IMAGE $APP_HOME/build/libs/$ARTIFACT_NAME .
	
	# Run
	EXPOSE 8080
	ENTRYPOINT exec java -jar ${ARTIFACT_NAME}
\end{Verbatim}

\newpage
\section{docker-compose.yml}
\begin{Verbatim}[numbers=left,xleftmargin=1cm,firstnumber=1,breaklines=true,tabsize=2]
	version: "3.3"
	services:
		node:
			container_name: "stipula-node"
			image: stipula-node:v0.4.2
			ports:
				- 127.0.0.1:8080:8080
				- 127.0.0.1:61000:61000
			volumes:
				- stipula-storage:/usr/app/storage/
			environment:
				- SEED=no

	volumes:
		stipula-storage:
\end{Verbatim}

If you don't want to manually build the Docker image, you can replace line 5 with 
\verb|image: "ghcr.io/federicozanardo/stipula-node:v0.4.2"|, this will download a specific 
image from a specific GitHub page \autocite{site:stipula-github-available-packages}.
