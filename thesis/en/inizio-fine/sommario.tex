% !TEX encoding = UTF-8
% !TEX TS-program = pdflatex
% !TEX root = ../tesi.tex

%**************************************************************
% Sommario
%**************************************************************
\cleardoublepage
\phantomsection
\pdfbookmark{Sommario}{Sommario}
\begingroup
\let\clearpage\relax
\let\cleardoublepage\relax
\let\cleardoublepage\relax

\chapter*{Abstract}

The purpose of this thesis is to provide a possible implementation of the domain-specific language 
\textit{Stipula}. The purpose of this language is to assist professionals such as lawyers in programming 
\textit{legal contracts}. The language is based on a set of programming abstractions that correspond to the 
distinctive elements that make up a legal contract, namely, permissions, prohibitions, obligations, asset 
transfer, and openness to external context, and that is likely to be executed on both centralized and 
distributed systems.\\
The unique characteristics of the language have strongly driven the implementation of a particular 
architecture, which is a solution obtained from a combination of different approaches present in the 
current blockchain context.\\
In parallel with the development of a contract execution system, the aim of the project is also to provide 
secure mechanisms for \textit{asset transfer}, to ensure that the transferred sum is not altered during 
the sending and receiving of a transaction, and avoid attacks such as \textit{double-spending}.\\
The implementation illustrated in this thesis is the first version that provides the general idea of the 
project, oriented towards the execution in different types of distributed systems, ranging from a simple 
client-server system to a network of replicated partially trusted nodes. To illustrate the implementation, 
the thesis fully discusses the case of non-trivial contracts such as the trading of assets and the rent of 
a bike. The final part of the paper presents the missing functionalities, addresses the current limits of 
the version presented, possible solutions proposed, and possible future developments of the project are 
introduced. In particular, for this last point, future developments are understood as evolutions from the 
point of view of security, usability, computation, and efficiency for the execution of legal contracts in 
distributed systems.

\endgroup			

\vfill
